\documentclass[10pt]{article}
\usepackage{graphicx}
\usepackage{geometry}
\usepackage{indentfirst}
\usepackage{amsmath}
\usepackage{amssymb}
\usepackage{booktabs}
\usepackage{setspace}
\usepackage{listings}
\usepackage{subfig}
\usepackage{hyperref}

\usepackage{color}
\usepackage[table]{xcolor}
\usepackage[nottoc,notlot,notlof]{tocbibind}

\definecolor{mygreen}{rgb}{0,0.6,0}
\definecolor{mygray}{rgb}{0.5,0.5,0.5}
\definecolor{mymauve}{rgb}{0.58,0,0.82}

\geometry{
    letterpaper,
    left = 0.75in,
    right = 0.75in,
    top = 30mm,
    bottom = 30mm
}

\lstset{ %
  backgroundcolor=\color{white},   % choose the background color; you must add \usepackage{color} or \usepackage{xcolor}
  basicstyle=\footnotesize,        % the size of the fonts that are used for the code
  breakatwhitespace=false,         % sets if automatic breaks should only happen at whitespace
  breaklines=true,                 % sets automatic line breaking
  captionpos=b,                    % sets the caption-position to bottom
  commentstyle=\color{mygreen},    % comment style
  deletekeywords={...},            % if you want to delete keywords from the given language
  escapeinside={\%*}{*)},          % if you want to add LaTeX within your code
  extendedchars=true,              % lets you use non-ASCII characters; for 8-bits encodings only, does not work with UTF-8
  frame=single,                    % adds a frame around the code
  keepspaces=true,                 % keeps spaces in text, useful for keeping indentation of code (possibly needs columns=flexible)
  keywordstyle=\color{blue},       % keyword style
  language=C++,                 % the language of the code
  morekeywords={*,...},            % if you want to add more keywords to the set
  numbers=left,                    % where to put the line-numbers; possible values are (none, left, right)
  numbersep=5pt,                   % how far the line-numbers are from the code
  numberstyle=\tiny\color{mygray}, % the style that is used for the line-numbers
  rulecolor=\color{black},         % if not set, the frame-color may be changed on line-breaks within not-black text (e.g. comments (green here))
  showspaces=false,                % show spaces everywhere adding particular underscores; it overrides 'showstringspaces'
  showstringspaces=false,          % underline spaces within strings only
  showtabs=false,                  % show tabs within strings adding particular underscores
  stepnumber=2,                    % the step between two line-numbers. If it's 1, each line will be numbered
  stringstyle=\color{mymauve},     % string literal style
  tabsize=2,                       % sets default tabsize to 2 spaces
  title=\lstname                   % show the filename of files included with \lstinputlisting; also try caption instead of title
}

\newcommand{\beac}{\textbf{Beacon }}
\newcommand{\doc}[1]{Dr.\@ #1}

\captionsetup{belowskip=12pt,aboveskip=12pt}

% \textit{blastp}
% Intel \textsuperscript{\textregistered} Xeon\textsuperscript{\textregistered} 
%  Intel \textsuperscript{\textregistered} Xeon Phi\texttrademark \hspace*{2pt}
% Intel \textsuperscript{\textregistered} VTune\texttrademark \hspace*{2pt}

% \begin{figure}[!htbp]
% \centering
% \includegraphics[scale=0.4]{../Images/Y1Q3+_IMAGES/hpc_blastp_speedup_host.eps}
% \caption{Speedup of \textit{blastp} on the Xeon processor using the best runtimes.}
%\end{figure}

\begin{document}

\title{HPC BLAST: User Manual\\
         }
\author{Application Acceleration Center of Excellence (AACE) and\\
   Intel Parallel Computing Center (IPCC) at the\\
   Joint Institute for Computational Sciences\\University of Tennessee}
  %\texttt{shane-sawyer@tennessee.edu}}
\date{\today}
\maketitle

\newpage

\tableofcontents

\newpage

%\listoffigures

%\newpage

\section{Introduction} \label{sec:intro}

This document is intended to be a user manual that demonstrates how to compile and execute HPC-BLAST, the high performance implementation of NCBI BLAST developed at the Intel Parallel Computing Center at the
University of Tennessee, Knoxville. Since HPC-BLAST compilation will happen as a result of compiling NCBI BLAST, this document also covers the compilation and execution of the reference implementation provided by NCBI.
The documentation also covers compilation and execution of mpiBLAST in order to provide a second parallel BLAST implementation for comparison purposes.  This document covers the compilation of of HPC-BLAST
for Intel Xeon and Intel Xeon Phi (Knights Landing) processors. Compilation for the AMD Epyc processor is also discussed.  The documentation also covers usage of auxiliary software tools commonly used in conjunction
with NCBI BLAST, tools specific to mpiBLAST, and tools specific to HPC-BLAST.  Finally, specific examples will be given to illustrate how to use the BLAST implementations, with priority given to the use of HPC-BLAST.
Note, many of the provided usage examples will include commands and tools that are specific to the software environment of the Joint Institute for Computational Sciences.

A special thanks is due to Dr. Albert Golembiowski of Intel who provided excellent documentation for compiling NCBI BLAST version 2.2.30.  The approach Dr. Golembiowski outlined is used as a basis for the compilation
guide of BLAST version 2.7.1 provided in this document.

\subsection{HPC-BLAST Overview} \label{ssec:hpc-overview}

This section is intended to quickly describe the programming model of HPC-BLAST in order to understand how HPC-BLAST allocates work to both threads and processes.  HPC-BLAST utilizes the Message Passing Interface
(MPI) to distribute search tasks between processes and OpenMP to parallelize searches within each process.  At the process level, the central idea is that of a replication group.  A replication group is simply a collection of MPI
processes (ranks) that each own a subset of the complete database.  These database subsets are produced by the NCBI tool \emph{makeblastdb}, included as part of the BLAST toolkit, which partitions the entire set of subject
sequences.  For a given replication group, each process owns different database components.  Within this replication group, each process is given the same set of queries to scan.  It is also possible to have a single rank comprise
a replication group.  In this case, the single rank searches its set of queries against the entire database.  HPC-BLAST allows multiple replication groups to be launched so that different replication groups may search only a portion
of the entire query input file.  The ranks in an HPC-BLAST parallel job also support multithreading.  The distribution of work within the rank using threads is similar to how work is distributed at the MPI level.   There are three different
classifications for the threads that HPC-BLAST can spawn; thread group, team leader, and search thread.  Further, all HPC-BLAST ranks in a parallel job have the same distribution of threads.

The thread classifications are now briefly described.  A thread group is a collection of threads, similar to replication groups, where each member of the thread group owns a portion of the database component assigned to the process.
As in the case of replication groups, threads in a thread group, called team leader threads, search with the same set of queries.  Thus, team leader threads in different thread groups search with different portions of the query set
assigned to the MPI process spawning the threads. The team leader threads are also able to spawn search threads.  The search threads are the threads that are provided by the NCBI BLAST toolkit.  The number of search threads
is given on the command line as is done with NCBI BLAST.  The particular distribution of threads into the different classifications is done at runtime by creation of a file and command line specification.  Examples are given in Section
\ref{ssec:runncbi}.  If NCBI-compliance is important for your work, the HPC-BLAST development team recommends using only thread groups and search threads in the thread distribution of a parallel job.  Here, NCBI-compliance
means that the results produced by HPC-BLAST will match the results from NCBI BLAST up to dependency in the order in which the queries were searched.  Experiments have shown that the use of team leaders in the thread
distribution leads to HPC-BLAST finding more matches than NCBI BLAST.  If finding additional matches is not problematic, you may find that using some team leaders leads to better performance than using thread groups and
search threads alone.

In the previous release of HPC-BLAST, version 1.0, it was observed that the best performance of HPC-BLAST was seen when the thread distribution favored thread groups with only 2-3 threads reserved for NCBI
search threads. With the move to NCBI BLAST toolkit version 2.7.1, the dynamic has changed due to improved parallelization of searching in the NCBI engine. In many cases, it may be best to allocate the available threads
to search threads. 

\clearpage

\section{HPC-BLAST and Tools Source Location} \label{sec:src}

For the most up-to-date version of HPC-BLAST, 
and associated auxiliary tools we recommend that you
download from our Git repository. This can be accomplished via
cloning the repository from the command line, or by downloading a zip
from our GitHub page.   

\begin{itemize}
  \item Git Repository Clone:
  \begin{itemize}
    \item Use the following command to clone HPC-BLAST to your machine:
      \item \verb|>$ git clone https://github.com/UTennessee-JICS/HPC-BLAST|
      \item Once cloned, you can update the code to the newest version using the following command (when in the HPC-BLAST directory):
      \item \verb|>$ git pull|
    \end{itemize}
  \item Git Zip Download:
  \begin{itemize}
    \item  Simply use the``zip download'' option on our webpage at:
    \item https://github.com/UTennessee-JICS/HPC-BLAST
  \end{itemize}
\end{itemize}


%The source code for HPC-BLAST and associated auxiliary tools are kept in the AACE HPC-BLAST GitHub repository.  
%You can get the souce code in one of 2 ways: (1) The sources for all auxiliary tools associated with HPC-BLAST, and described in Chapter \ref{sec:auxtools}, can be found at:
%\begin{verbatim}
%/PATH/TO/SVN/codes/HPC-BLAST/AUXILIARY-TOOLS
%\end{verbatim}

\noindent The sources for all auxiliary tools associated with HPC-BLAST, and described in Chapter \ref{sec:auxtools}, can be found at:
\begin{verbatim}
/PATH/TO/REPO/AUXILIARY-TOOLS
\end{verbatim}

\noindent The list of files needed for the auxiliary tools are: \emph{database\textunderscore sample.c}, \emph{distribute\textunderscore queries.cpp}, \emph{get\textunderscore query\textunderscore stats.c}, \emph{stitch\textunderscore blast\textunderscore output.cpp}, \emph{strip\textunderscore header.c},
\emph{compare\textunderscore results.c}, \emph{SFMT.c}, \emph{SFMT-sse2.h}, \emph{SFMT-params.h}, \emph{SFMT-params607.h}, and \emph{compile.dat}.  A makefile is included in the directory
to build the binaries. It assumes that the Intel compiler suite is available.


\clearpage

\section{Compiling BLAST} \label{sec:compile}

This section of the user manual focuses on the compilation strategies employed to successfully build working binaries of the BLAST implementations for both Intel Xeon and Xeon Phi processors.
The example build guides assume the use of the Intel compilers and the Intel MPI libraries, specifically Intel Parallel Studio version 2017.2.174.  Version 6.3.0 of GCC was also used for needed system libraries and tools.
The build guides also provide suggested make
flags that have been tested to work; others combinations may be possible, but have not been tested.\\

\noindent {\bf NOTE: }An experimental script, \verb^make_hpc_blast.sh^, is provided in the repository that implements the instructions outlined below. It is intended to automate the process and may
work out of the box for you, but it is likely that some commands will need to be modified for your particular setup. \\

\subsection{Compiling NCBI/HPC BLAST v2.7.1} \label{ssec:ncbi-compile}

The repository includes a copy of the tarball obtained from NCBI.
Uncompress the tar file:
\begin{verbatim}
$> tar xvzf ncbi-blast-2.2.31+-src.tar.gz
\end{verbatim}
\noindent and enter the top level directory of the toolkit:
\begin{verbatim}
$> cd ncbi-blast-2.2.31+-src/c++
\end{verbatim}
\noindent Next, we copy the files for HPC-BLAST into the appropriate directories of the NCBI BLAST toolkit.  
For simplicity, we will assume that a local copy of the
GitHub repository is available for copying the files.  
\footnotesize
\begin{verbatim}
$> cp /PATH/TO/REPO/api/blast_options_handle.cpp src/algo/blast/api
$> cp /PATH/TO/REPO/api/local_blast.cpp src/algo/blast/api
$> cp /PATH/TO/REPO/api/traceback_stage.cpp src/algo/blast/api

$> cp /PATH/TO/REPO/include/local_blast.hpp include/algo/blast/api
$> cp /PATH/TO/REPO/include/traceback_stage.hpp include/algo/blast/api

$> cp /PATH/TO/REPO/blast/hpc_blastp_app.cpp src/app/blast
$> cp /PATH/TO/REPO/blast/hpc_blastn_app.cpp src/app/blast
$> cp /PATH/TO/REPO/blast/Makefile.hpc_blastp.app src/app/blast
$> cp /PATH/TO/REPO/blast/Makefile.hpc_blastn.app src/app/blast
$> cp /PATH/TO/REPO/blast/Makefile.in src/app/blast
\end{verbatim}
\normalsize

\noindent \rule{16cm}{0.4pt}

\noindent Now, we can begin the compilation.  Enter the  \verb^c++^ directory.  The first step is to set up the environment variables:
\begin{verbatim}
$> export CC=mpiicc
$> export CXX=mpiicpc
$> export LD=xild
$> export AR="xiar crs" 
\end{verbatim}
\noindent Depending on your system's configuration, you may need to execute the Intel compiler suite script to expose the compilers.  Alternatively, the full path to the compilers can be substituted.  Finally, note that we are explicitly using the MPI variants
of the Intel compilers in order to build HPC-BLAST during the NCBI BLAST build.  Next, the configuration script is executed:
\begin{verbatim}
$> ./configure --without-3psw --with-bin-release --with-static-exe --with-mt \
 --without-debug --without-boost --without-strip --with-build-root=< ... >
\end{verbatim}
\noindent where \verb^< ... >^ is the name of the desired build directory and is relative to the \verb^c++^ directory.  Specifying the root directory of the build is optional.  If left off, the configure script will create a default directory labeled \verb^ReleaseMT^, possibly
prepended with the compiler version.  Now, enter the build directory of the root directory:
\begin{verbatim}
$> cd ROOT-DIR/build
\end{verbatim}
\noindent Next, the file \verb^Makefile.mk^ is to be edited. The instructions are:
\begin{itemize}
\item Add \verb^-ww3377^ to the end of lines 83, 84, and 86 corresponding to \verb^CONF_C^, \verb^CONF_CXX^, and \verb^CONF_CXXCPP^.
\item Change the \verb^-O2^ flag on lines 93 and 94 corresponding to \verb^CONF_CFLAGS^ and \verb^CONF_CXXFLAGS^ to \verb^-O3^.
\item Change the \verb^-O^ to \verb^-O3^ on line 98 corresponding to \verb^CONF_LDFLAGS^.
\item Add the appropriate flag on lines 235-237 to specify architectural specific instructions should be used for files compiled with ``FAST'' flags. For instance, if compiling for the Intel Xeon Phi (Knights Landing), use
\verb^-xMIC-AVX512^.
\end{itemize}
\noindent Now, the build process can begin with
\begin{verbatim}
$> make all_r
\end{verbatim}
\noindent The compilation may require a few hours depending on processor load and system configuration.  The process may be dramatically sped up using multiple cores.  For example, if we were to compile on a typical compute
node, we might instead use:
\begin{verbatim}
$> make -j8 -k all_r 
\end{verbatim}
\noindent where \verb^-j8^ instructs the compiler to use up to 8 cores in parallel and \verb^-k^ instructs the compiler to ignore warning and continue compiling.  After completing, the binaries, including HPC-BLAST, will be placed in:
\begin{verbatim}
./ROOT-DIR/bin
\end{verbatim}

\subsection{Compile Notes for NCBI/HPC BLAST v2.7.1 on the AMD Epyc} \label{ssec:ncbi-compile}

HPC-BLAST 1.1.0 can be compiled for the AMD Epyc, as well. The procedure is similar to the one outlined above for the Intel Xeon and Xeon Phi processors. HPC-BLAST has been compiled successfully using
Intel Parallel Studio 2017.2.174 and GCC 6.3.0. Before running the configuration, set up the environment as follows:
\begin{verbatim}
$> export CC=mpicc
$> export CXX=mpicxx
$> export LD=ld
$> export AR=ar 
\end{verbatim}
This way the Intel MPI libraries will be linked, but the sources will be compiled with GCC. Now, run the configuration script as above. Note that when editing the file Makefile.mk,
it is no longer necessary to use the flag \verb^-ww3377^. For building architecture specific binaries, use the flag
\verb^-march=znver1^ on lines 235-237 of the Makefile.mk file.

\subsection{Compiling mpiBLAST} \label{sec:mpicomp}

The source code for mpiBLAST version 1.6.0 can downloaded from:
\begin{verbatim}
http://www.mpiblast.org/Downloads/Stable
\end{verbatim}
\noindent we may download it directly using:
\begin{verbatim}
$> wget http://www.mpiblast.org/downloads/files/mpiBLAST-1.6.0.tgz
\end{verbatim}
\noindent Unpack the compressed file into a convenient location:
\begin{verbatim}
$> tar xvzf mpiBLAST-1.6.0.tgz
\end{verbatim}

\subsubsection{Compiling for Intel Xeon processors} \label{sssec:mpiblasthost}
\noindent Enter the top level directory of the mpiBLAST source directory,
\begin{verbatim}
$> cd /PATH/TO/mpiblast-1.6.0
\end{verbatim}
\noindent As in the compilation of HPC-BLAST, mpiBLAST requires that NCBI BLAST is compiled first.  The appropriate NCBI BLAST version is already packaged with the mpiBLAST tarball.
From the top level directory, go into the \verb^ncbi/platform^ directory.
\begin{verbatim}
$> cd ncbi/platform
\end{verbatim}
\noindent Edit the file \verb^linux_icc.ncbi.mk^ by changing, on line 8, \verb^NCBI_AR=ar^ to \verb^NCBI_AR=xiar^.  On lines 11 and 12,
remove the option \verb^-tpp7^ from the two variables \verb^NCBI_LDFLAGS1^ and \verb^NCBI_OPTFLAG^.  Next, go up one level in
the directory structure and go into the \verb^make^ directory.
\begin{verbatim}
$> cd ../make
\end{verbatim}
\noindent Then, edit the file \verb^makedis.csh^ by changing, on line 316, from
\begin{verbatim}
set NCBI_DOT_MK = ncbi/platform/${platform}.ncbi.mk
\end{verbatim}
\noindent to
\begin{verbatim}
set NCBI_DOT_MK = ncbi/platform/linux_icc.ncbi.mk
\end{verbatim}
\noindent Finally, uncomment lines 358 and 359 which set both \verb^HAVE_OGL^ and \verb^HAVE_MOTIF^ to 0.
These changes will ensure that the NCBI BLAST build uses the Intel compilers to build the binaries.  Now, move up two directory levels to the root directory and start the build process with:
\begin{verbatim}
$> ./ncbi/make/makedis.csh
\end{verbatim}
\noindent Next, we compile mpiBLAST.  Set up the appropriate environment variables:
\begin{verbatim}
$> export CC=icc
$> export CXX=icpc
$> export AR=xiar
\end{verbatim}
\noindent Then, run the configure script:
\begin{verbatim}
$> ./configure
\end{verbatim}
\noindent Next, open the make file in the \verb^src^ subdirectory, e.g.:
\begin{verbatim}
$> emacs src/Makefile
\end{verbatim}
\noindent Make the following changes to this make file:
\begin{itemize}
\item Adjust the compiler on line 108 from \verb^mpicc^ to \verb^mpiicc^ to invoke the Intel MPI compiler.
\item Change the optimization level on line 110 from \verb^-O2^ to \verb^-O3^.  Optionally, the \verb^-g^ flag may be removed.
\item Adjust the compiler on line 118 from \verb^mpicxx^ to \verb^mpiicpc^ to invoke the Intel MPI compiler.
\item Change the optimization level on line 120 from \verb^-O2^ to \verb^-O3^.  Optionally, the \verb^-g^ flag may be removed.
\end{itemize}
\noindent Finally, begin the compilation process with \verb^make^.
\begin{verbatim}
$> make
\end{verbatim}
\noindent The binaries (\verb^mpiformatdb^, \verb^mpiblast^, and \verb^mpiblast_cleanup^) will be placed in the \verb^src^ subdirectory upon completion.

\clearpage

\section{Auxiliary Tools} \label{sec:auxtools}

This section introduces the use of various tools that are either packaged with NCBI BLAST or are part of the HPC-BLAST toolchain.  For tools included as part of the NCBI BLAST distribution, refer
to the NCBI website (\url{http://www.ncbi.nlm.nih.gov/books/NBK1763/}) for complete instructions.

\subsection{\emph{makeblastdb}} \label{ssec:makeblastdb}

\noindent {\bf Note: } This tool is compiled as part of the NCBI BLAST toolkit.\\

The \verb^makeblastdb^ tool is built as part of the NCBI BLAST compilation and is used to format sequence databases from FASTA format to the appropriate BLAST format used during searches.  The simplest example of
use is:
\begin{verbatim}
$> ./makeblastdb -in <DB file> -dbtype <TYPE>
\end{verbatim}
\noindent where \emph{DB file} is the name of the FASTA database and \emph{TYPE} is either \emph{prot} or \emph{nucl} to indicate protein or nucleotide sequences, respectively.  If the database name is \emph{fasta-db} and
consists of proteins, the output will be \emph{fasta-db.psq}, \emph{fasta-db.phr}, and \emph{fasta-db.pin}.  The three output files correspond to the sequence file, header information file, and the index file for accessing the other files.
If the file contained nucleotides instead, the three output files would be of extensions \emph{nsq}, \emph{nhr}, and \emph{nin} instead.  The previous example also assumes that the database is small enough, i.e. the index file does
not exceed 2GB in size.  If instead the database is such that the index file exceeds 2GB, the database of subject sequences is partitioned.  Each subset of the partition has the three files (sequence, header, and index) and are labeled by an index starting at \(00\)
and going to \(P-1\), where \(P\) is the number of subsets in the partition.  Additionally, an alias file is created: either as \emph{fasta-db.pal} or \emph{fasta-db.nal}.  The alias file lists all the indices of the subsets in the set so that BLAST can cycle
through the partition one subset at a time.\\

\noindent The \verb^makeblastdb^ tool can also take the additional argument \verb^-max_file_sz^ to limit the maximum file size of any file in the partition.  The maximum file size may be specified in bytes or higher units: e.g. 1000000000B, 
1000000KB, 1000MB, or 1GB.  This is particularly useful for HPC-BLAST as this provides a natural mechanism for breaking up a database into smaller constituent subsets for distribution between MPI ranks.  Through trial and error of the
appropriate maximum file size, a reasonably balanced partitioning of the database can be generated.  When doing so, keep in mind that it is the \emph{.[np]sq} files that should be balanced as these contain the actual subject sequence contents.

\subsection{\emph{blastdbcmd}} \label{ssec:blastdbcmd}

\noindent {\bf Note: } This tool is compiled as part of the NCBI BLAST toolkit.\\

The \verb^blastdbcmd^ tool is useful in that it allows the conversion from BLAST formatted files into FASTA format.  Suppose we have a BLAST database that we wish to examine in FASTA format.  The database files are
\emph{fasta-db.psq}, \emph{fasta-db.phr}, and \emph{fasta-db.pin}.  Then, to convert them to FASTA format, we would issue:
\begin{verbatim}
$> ./blastdbcmd -db fasta-db -dbtype prot -entry all -out <file name> -outfmt %f
\end{verbatim}

\noindent \verb^blastdbcmd^ is also useful for querying BLAST formatted databases to determine the number of sequences, longest sequence, and counting the total number of residues.  This is done with:
\begin{verbatim}
$> ./blastdbcmd -db <db name> -dbtype <residue type> -info
\end{verbatim}
\noindent where the residue type is either \emph{prot} or \emph{nucl}.  In the case of a database that is partitioned, the database file name can be either the name of the set, such as \emph{nr}, or a subset of the partition,
such as \emph{nr.00}.\\

The \verb^-info^ option is particularly important for HPC-BLAST when employing a distributed database approach; see Section \ref{ssec:runhpc} for examples. In order for HPC-BLAST to compute the correct E-values, the total number of
residues must be known and passed on the command line.  For example, suppose there is a BLAST formatted protein database called \emph{seq-db} that has been partitioned into 8 subsets. The total number of residues can be see in the output from:
\begin{verbatim}
$> ./blastdbcmd -db seq-db -dbtype prot -info
Database: .seq-db.fasta
	5,815,854 sequences; 2,006,616,771 total residues

Date: Jun 1, 2014  9:00 PM	Longest sequence: 41,943 residues

Volumes:
	/PATH/TO/DB/seq-db.00
	/PATH/TO/DB/seq-db.01
	/PATH/TO/DB/seq-db.02
	/PATH/TO/DB/seq-db.03
	/PATH/TO/DB/seq-db.04
	/PATH/TO/DB/seq-db.05
	/PATH/TO/DB/seq-db.06
	/PATH/TO/DB/seq-db.07
\end{verbatim}
\noindent Here, the database contains 2,006,616,771 residues.\\

\noindent {\bf Note:} The above example assumes that the alias file exists and has not been renamed so that each subset can be read in one at a time. If the alias file has been deleted, the operation can still be performed, but all database subsets
of the partition will have to be listed on the command line in the following manner:
\begin{verbatim}
$> ./blastdbcmd -db "seq-db.00 seq-db.01 seq-db.02 seq-db.03 seq-db.04 \
  > seq-db.05 seq-db.06 seq-db.07" -dbtype prot -info
\end{verbatim}

\subsection{\emph{strip\textunderscore header}} \label{sssec:stripheader}

\noindent {\bf Note: } This tool is part of the HPC-BLAST toolchain and needs to be manually compiled.  Please refer to Chapter \ref{sec:src}.\\

The \verb^strip_header^ tool is used to remove excess sequence identifiers from FASTA formatted databases.  For the \emph{nr} database, the sequence identifier lines can be several thousand characters long.  This results in having the
associated \emph{*.phr} files be as long or longer than the associated sequence data.  The tool can be used to cut off the identifier line after a specified length:
\begin{verbatim}
$> ./strip_header -i db.fasta -o db.mod.fasta -l XX
\end{verbatim}
\noindent where \verb^XX^ is the number of characters to retain and the \verb^-l^ is l as in length.

\subsection{\emph{database\textunderscore sample}} \label{ssec:datadist}

\noindent {\bf Note: } This tool is part of the HPC-BLAST toolchain and needs to be manually compiled.  Please refer to Chapter \ref{sec:src}.\\

The \verb^database_sample^ tool is used as part of testing the performance of various BLAST implementations.  It is largely designed to sample sequences from a FASTA database in order to generate a representative sample that
can be used as an input query file.  A basic example of usage is:
\begin{verbatim}
$> ./database_sample -i <input> -o <output> -s <x>
\end{verbatim}
\noindent where \verb^x^ is the number of sequences desired.  There are additional optional parameters that can give a more refined search.  For example, using
\begin{verbatim}
$> ./database_sample -i <input> -o <output> -s <x> -f 1000
\end{verbatim}
\noindent will cause the program to attempt to sample \verb^x^ sequences with length \verb^1000^.  If there are fewer than \verb^x^ sequences of this length, an attempt is made to search within a narrow band around the specified length; i.e., the specified
length plus or minus some tolerance.
We can also provide a ranged search using the \verb^-b^ and \verb^-e^ arguments to supply a beginning and ending range from which to sample.  These arguments can be used together or separately.  If only one is used, then the minimum or maximum
sequence length is used as the other parameter.  Lastly, the total number of residues desired in the output can be specified with \verb^-t^.  Using this argument, sampling will terminate once the total number of residues has hit or exceeded the
desired amount.\\

\noindent As a final example, suppose we wished to generate a query file consisting of 20,000 residues with all sequences between 500 and 2000 residues in length from the file \emph{db.fasta}.  This can be accomplished by:
\begin{verbatim}
$> ./database_sample -i db.fasta -o query.fasta -s 500 -b 500 -e 2000 -t 20000
\end{verbatim}
\noindent It is typical to over estimate the number of sequences in the output when specifying a total number of residues, especially when the sequences are sampled over a range of lengths.  This is done to ensure that we have enough sequences
to write to the query file.  When both \verb^-s^ and \verb^-t^ are specified on the command line, the \verb^-t^ option will take precedence over \verb^-s^ and sampling will stop once the desired number of letters (residues) has been achieved.

\subsection{\emph{distribute\textunderscore queries}} \label{ssec:distqueries}

\noindent {\bf Note: } This tool is part of the HPC-BLAST toolchain and needs to be manually compiled.  Please refer to Chapter \ref{sec:src}.\\

In Section \ref{ssec:datadist}, the construction of a query file by sampling a FASTA database is shown.  The generated query file is sufficient for use with NCBI BLAST and other implementations such as mpiBLAST.  For HPC-BLAST, a second step is required; at
least with the current version.  The next step is to distribute the queries for the different replication groups and thread groups within each rank.  This is accomplished with the \verb^distribute_queries^ tool.  \verb^distribute_queries^ reads in the query file and then distributes
the queries evenly between all replication groups.  Within each replication group, the queries are distributed to the thread groups using a simple greedy algorithm.  Each thread group is associated with a bucket.  At the start of the distribution, all buckets are empty and
the queries are sorted by descending length.  Next, the code loops over the queries and assigns a query to the bucket with the least residues in it.  If after adding a new sequence, the number of residues in a bucket exceeds a threshold, the bucket is marked as full
and stops taking new sequences.  If all buckets become full and there are still sequences to distribute, the buckets are reset to empty (i.e. number of residues is set to 0) and the process continues.  The value of the threshold is dependent on whether the
sequences are amino acids or nucleotides.  The usage of the application is:
\begin{verbatim}
$> ./distribute_queries -i <FASTA input> -r X -g Y -t Z -o <output name>
\end{verbatim}
\noindent where \verb^X^ is the number of replication groups in the parallel job, \verb^Y^ is the number of thread groups each rank utilizes, and \verb^Z^ is either 'a' for amino acids or 'n' for nucleotides.  The output name is optional.  If absent, the input file name
will be used for the output files.  Output files are generated for each thread group and each replication group.  For example, consider a parallel job with 4 replication groups and 32 thread groups per rank.  The name of the input file is \emph{query.prot}. 
We would then use:
\begin{verbatim}
$> ./distribute_queries -i query.prot -r 4 -g 32 -t a
\end{verbatim}
\noindent This would result in output files identified by \emph{query.prot.x.y}, where \emph{x} (ranges 0..3) corresponds to the replication group and \emph{y} (ranges 0..31) corresponds to the thread groups.\\

As a final note, \verb^distribute_queries^ also produces the file \emph{query\textunderscore ids.unsorted}.  This file is used to recombine all output from all ranks in the parallel job into the correct order corresponding to the original query ordering before
distribution.

\subsection{\emph{get\textunderscore query\textunderscore stats}} \label{ssec:getqstats}

\noindent {\bf Note: } This tool is part of the HPC-BLAST toolchain and needs to be manually compiled.  Please refer to Chapter \ref{sec:src}.\\

The \verb^get_query_stats^ tool can be used to get high level information on a FASTA file, including the number of sequences, maximum/minimum sequence length, average sequence length, and total number of residues.  It can be used as:
\begin{verbatim}
$> ./get_query_stats query-file.fasta
\end{verbatim}

\noindent {\bf Note:} \verb^get_query_stats^ can be used on database files in FASTA format to determine the number the number of residues. Here, the output will specify the total number of letters; letters is synonymous with residues.

\subsection{\emph{stitch\textunderscore blast\textunderscore output}} \label{ssec:stitch}

\noindent {\bf Note: } This tool is part of the HPC-BLAST toolchain and needs to be manually compiled.  Please refer to Chapter \ref{sec:src}.\\

The \verb^stitch_blast_output^ tool is used to recombine all the output files generated from an HPC-BLAST run into a single, merged output file.  This requires that both the \emph{query\textunderscore ids.unsorted} file from the distribution step and the
\emph{job\textunderscore params} file from the job configuration step (see Section \ref{ssec:runhpc}) are present in the directory where the HPC-BLAST
output files are located so that the correct order of sequences can be established.  It is used as:
\begin{verbatim}
$> ./stitch_blast_output <HPC-BLAST output name> <optional new output name>
\end{verbatim}
\noindent where the HPC-BLAST output name is the name given during the execution of the search.\\

\noindent {\bf Note: } The  \verb^stitch_blast_output^ tool only supports output files written with the pairwise format. This is the default behavior of NCBI-BLAST when no output format is specified on the command line. HPC-BLAST has only been tested with the pairwise output
but should operate correctly with other output formats. However, there is no supported way of merging the various files written by HPC-BLAST if another output format is desired.

\subsubsection{Special Note for Stitching Output} \label{sssec:stitch-note}

When merging files from parallel jobs that used database decomposition, it is likely that \verb^stitch_blast_output^ will generate errors stating that a subject match to a particular query has a 0 length alignment to write to the merged output.
This situation arises when there are many subject matches that all have comparable E-values and Bit Scores appearing at, or near, the cutoff for alignment output (by default, NCBI BLAST will output 500 description lines but only 250 alignments).
The nature of the problem is that there is not, in general, sufficient floating point precision to fully sort and merge the results.

\subsection{\emph{compare\textunderscore results}} \label{ssec:compres}

\noindent {\bf Note: } This tool is part of the HPC-BLAST toolchain and needs to be manually compiled.  Please refer to Chapter \ref{sec:src}.\\

The \verb^compare_results^ tool is used to quantitatively evaluate the differences in sequence alignments between different implementations and versions of BLAST.  It reads in two BLAST output files and examines the alignments found for each
query.  Currently, the tool only supports the ASCII pairwise format that is the default for BLAST.  It is used with:
\begin{verbatim}
$> ./compare_results -r <file1> -R <file2> -o <output>
\end{verbatim}

\clearpage

\section{Running BLAST} \label{sec:execblast}

This section demonstrates the basic mechanisms for executing BLAST searches using a variety of implementations.  This section also assumes that the binaries have all been built for the appropriate architectures and general familiarity with the tools described
in previous sections.

\subsection{Running NCBI BLAST} \label{ssec:runncbi}

This section describes the basic steps of executing a BLAST search using the NCBI implementation.  NCBI BLAST offers a large number of options and parameters that are not covered in this document. Refer to the NCBI website
at \url{http://www.ncbi.nlm.nih.gov/books/NBK1763} for a more thorough introduction.
It is assumed that both a BLAST formatted database and query file have been created.\\

\noindent To run a protein search with NCBI BLAST, we can execute with:
\begin{verbatim}
$> ./blastp -db <db name> -query <query file> -num_threads <x> -out <file name>
\end{verbatim}
\noindent where \verb^x^ is an integer specifying the number of threads to be used by the BLAST engine.\\

\noindent To run a nucleotide search with NCBI BLAST, we can execute with:
\begin{verbatim}
$> ./blastn -task blastn -db <db name> -query <query file> -num_threads <x> /
-out <file name>
\end{verbatim}
\noindent Here, it is important to use \verb^-task blastn^ in order to use the canonical BLAST search for nucleotides.  This is particularly important for versions 2.2.29 and above of NCBI BLAST since this option is necessary to use the contributions made
by Dr. A. Golembiowski.

\subsection{Running HPC-BLAST} \label{ssec:runhpc}

This section demonstrates the steps to run HPC-BLAST.  Several examples are presented to illustrate how HPC-BLAST is used in a variety of contexts. The examples include a single rank job, multiple ranks
each sharing the database, and multiple ranks with replication groups.

\subsubsection{Single Rank Example} \label{sssec:hpcsingle}

The first step is to determine how many threads are to be used in the HPC-BLAST run.  For example, say we are using an Intel Xeon processor and want to use a total of 60 threads.  Next, we determine how we wish the threads to be
distributed in the single rank.  Suppose we want to there to be 4 thread groups, 5 team leaders per thread group, and each team leader will use a total of 3 BLAST search threads (\(4 \times 5 \times 3 = 60\)).  To express this for HPC-BLAST, we write the
parameters of the job into a file called \emph{job\textunderscore params}.  The \emph{job\textunderscore params} file is read by HPC-BLAST and by \emph{stitch\textunderscore blast\textunderscore output} and is expected to be in the current working directly.
For this example, we can express the parallel job configuration with:
\begin{verbatim}
$> echo 4 5 1 1 > job_params
\end{verbatim}
\noindent This means that HPC-BLAST will be executed with 4 thread groups, 5 team leaders per thread group, 1 rank per replication group, and 1 replication group.  Recall from Section \ref{ssec:hpc-overview} thread groups are used to distribute queries within a rank and team leaders are used to
distribute the database within a thread group.  The \emph{job\textunderscore params} must be in the directory where HPC-BLAST will be executed.  Next, the queries are distributed as shown in Section \ref{ssec:distqueries}.
\begin{verbatim}
$> ./distribute_queries -i query.fasta -r 1 -g 4 -t a
\end{verbatim}
\noindent Now, we can start the BLAST search.  For example, suppose we are using a compute node on the \beac cluster.  We can start the job with:
\begin{verbatim}
$> micmpiexec -n 1 /PATH/TO/WORKDIR/hpc_blastp \
-db db-name -query query.fasta -num_threads 3 -out blast.ouput
\end{verbatim}
\noindent After HPC-BLAST finishes execution, a single output file will be generated: \emph{blast.output.0.0.0.0.0}.

\subsubsection{Single Rank Replication Groups} \label{sssec:hpc1repg}

In the next example, we will utilize multiple replication groups on the same host.  Each replication group will use the same database; thus, only query distribution will occur at the MPI level.  Suppose we use 4 replication groups and wish to run on an Intel Xeon processor.  Suppose we are running on a Skylake model which has 20 cores, each supporting 2 threads for a total 40 logical cores.
So, each rank of HPC-BLAST should be limited to 10 threads, at most, to prevent over-saturating the hardware. We will limit ourselves to 32 threads in this example.
We then decide to use 2 thread groups with 2 team leaders and two search threads for each rank.  We write these parameters for the run into the \emph{job\textunderscore params} file:
\begin{verbatim}
$> echo 2 2 1 4 > job_params
\end{verbatim}
\noindent This means that HPC-BLAST will run with two thread groups and two team leaders per rank and 4 replication groups, each group comprised of a single rank.  Since each rank is its own replication group, the queries can be distributed across the MPI
ranks.  The queries are distributed with:
\begin{verbatim}
$> ./distribute_queries -i query.nucl -r 4 -g 2 -t n
\end{verbatim}
\noindent Next, we can start the job on a single node on a compute node:
\begin{verbatim}
$> micmpiexec -n 4 /PATH/TO/WORKDIR/hpc_blastn \
-task blastn -db db-name -query query.nucl -num_threads 2 -out blast.output
\end{verbatim}
\noindent After HPC-BLAST finishes execution, several files will be generated following the convention \emph{blast.output.x.x.0.0.0}, where  \emph{x} corresponds to the replication groups and ranges over \([0..3]\). 
Finally, the output created from the HPC-BLAST run can be merged into a single output file with:
\begin{verbatim}
$> ./stitch_blast_output blast.output
\end{verbatim}

\subsubsection{Multiple Ranks per Replication Group} \label{sssec:hpcrepgs}

In the following examples, replication groups consisting of multiple ranks are used.  In this situation, different ranks in the replication group(s) search across different databases or separate subsets of a database partition; the latter being the most common example of usage.
As seen in Section \ref{ssec:makeblastdb}, we can use \emph{makeblastdb} to partition a database into pieces that are roughly the same size.\\

In the first example, we will assume we have a database that has been partitioned into 4 subsets: \emph{db.00}, \emph{db.01}, \emph{db.02}, and \emph{db.03}.  As described in Section \ref{ssec:blastdbcmd}, we determine the number of residues in the
database:
\begin{verbatim}
$> ./blastdbcmd -db db -dbtype prot -info
Database: db.fasta
	679,231 sequences; 18,394,721 total residues

Date: Feb 4, 2015  11:00 AM	Longest sequence: 11,055 residues

Volumes:
	/PATH/TO/DB/db.00
	/PATH/TO/DB/db.01
	/PATH/TO/DB/db.02
	/PATH/TO/DB/db.03
\end{verbatim}

\noindent Since there are 4 subsets in the partition, we will need 4 ranks per replication group unless we want ranks to cycle over multiple databases.
Let us also use 4 replication groups for the parallel job.  Thus, we will launch with a total of 16 MPI ranks.  We will also use Intel Xeon processors and map a single rank to each compute node.  On the rank level, we will use 32
threads and configure them as follows: 16 thread groups with 2 team leaders using a single search thread.  Now, put these parameters into the \emph{job\textunderscore params} file:
\begin{verbatim}
$> echo 16 2 4 4 > job_params
\end{verbatim}
\noindent Next, distribute the queries:
\begin{verbatim}
$> ./distribute_queries -i query.prot -r 4 -g 16 -t a
\end{verbatim}
\noindent Assuming we have submitted a batch job or an interactive job with 16 compute nodes, we can start the parallel job with:
\begin{verbatim}
$> micmpiexec -n 16 /PATH/TO/WORKDIR/hpc_blastp \
-db db -query query.prot -num_threads 1 -out blast.output -dbsize 18394721
\end{verbatim}
\noindent {\bf Note:}  Recall that \emph{makeblastdb} creates an alias file when a FASTA database is partitioned into subsets.  This file should be deleted, or hidden, to prevent BLAST from attempting to cycle over the entire partition rather than use the database subset HPC-BLAST
will assign the rank.\\

\noindent During the execution of HPC-BLAST, each rank will determine which replication group it is in based on its MPI rank and the runtime parameters specified in \emph{job\textunderscore params}.  The ranks also determine which subset of the partition  they are responsible for and
read that database subset during the search process.  For example, rank 7 of the parallel job will be in the second replication group and will search the \emph{db.03} subset.\\

\noindent After HPC-BLAST finishes execution, several files will be generated following the convention \emph{blast.output.x.x.y.0.0}, where  \emph{x} corresponds to the replication groups and ranges over \([0..3]\), and \emph{y} corresponds to the rank's index within the replication group and ranges over \([0..3]\).
Finally, the output created from the HPC-BLAST run can be merged
into a single output file with:
\begin{verbatim}
$> ./stitch_blast_output blast.output
\end{verbatim}

\subsubsection{Checkpointing}\label{sssec:checkpointing}

If HPC-BLAST stops prior to completion, for any reason, a resubmit of the original job will restart each search (for every replication group, for every 
rank within that replication group, for each thread group within that rank, and for every team leader within that thread group, there is a search) at the 
correct position in the search.  As HPC-BLAST does its work, it keeps track of where it is, using one restart file per search, following 
the convention (using the previous configuration as an example) 
\emph{blast.output.x.x.y.0.0.restart}, where  \emph{x} corresponds to the replication groups and ranges over \([0..4]\), and \emph{y} corresponds to the rank's index in the replication group
and ranges over \([0..7]\).  Note that if HPC-BLAST stops prior to completion, and one desires to resubmit the same job, but from the beginning, the restart 
files must be manually deleted.

\subsection{Running mpiBLAST} \label{ssec:runmpi}

This section provides examples on how to use mpiBLAST.  It is not intended to be a complete guide to using mpiBLAST.  Refer to the official user guide at \url{http://www.mpiblast.org/Docs/Guide} for further instruction.  Also note that mpiBLAST requires a minimum
of three ranks in order to run.  One rank is the super master process, one rank is the master process, and the remaining rank is the worker process.\\

\noindent {\bf Note 1:}  mpiBLAST uses a modified \emph{.ncbirc} file to read information on where to find files.  The basic template of the \emph{.ncbirc} file for mpiBLAST is:
\begin{verbatim}
[NCBI]
Data=/PATH/TO/mpiblast-1.6.0/ncbi/data
[BLAST]
BLASTDB=/PATH/TO/DB/SPACE
BLASTMAT=/PATH/TO/mpiblast-1.6.0/ncbi/data
[mpiBLAST]
Shared=/PATH/TO/DB/SPACE
Local=/PATH/TO/LOCAL/STORAGE
\end{verbatim}
Note that \verb^Data^ and \verb^BLASTMAT^ paths match as do \verb^BLASTDB^ and \verb^Shared^.  The \verb^Shared^ path is where the database is expected to be for mpiBLAST execution.  The \verb^Local^
path is used to specify the directory of locally mounted storage, i.e. storage connected directly to compute nodes.  In the examples provided below, the use of the virtual database frags flag indicates that we do not
wish to load database fragments to local storage but instead load them into RAM.\\

In the first example, we will run mpiBLAST using 14 worker processes on a single compute node.  First, we need to format the database with the tool provided in the mpiBLAST distribution:
\begin{verbatim}
$> ./mpiformatdb -N 14 -i nr -o F -p T
\end{verbatim}
\noindent This will partition the \emph{nr} database into 14 subsets.  Next, we can launch the parallel job with:
\begin{verbatim}
$> micmpiexec -n 16 /PATH/TO/WORKDIR/mpiblast \
--partition-size=15 --use-parallel-write \
--use-virtual-frags -p blastp -d nr -i query.prot -a 1 -o mpiblast.out
\end{verbatim}
\noindent Note that we launched with 16 MPI ranks: 14 workers, one master, and one super master.  The option \emph{--partition-size=15} tells mpiBLAST that a partition has 14 workers (plus one for the master).  The options \emph{--use-parallel-write}
and \emph{--use-virtual-frags} are suggested by the mpiBLAST documentation for best performance.  Similarly, if we ran the same job configuration with nucleotides instead of proteins, we would use:
\begin{verbatim}
$> ./mpiformatdb -N 14 -i nt -o F -p F
\end{verbatim}
\noindent And:
\begin{verbatim}
$> micmpiexec -n 16 /PATH/TO/WORKDIR/mpiblast \
--partition-size=15 --use-parallel-write \
--use-virtual-frags -p blastn -d nt -i query.prot -a 1 -o mpiblast.out
\end{verbatim}
\noindent In the above cases, the \emph{-a} option is to tell BLAST how many threads to use in the search.\\

For a second example, suppose we want to again use 14 worker processes, but now wish to have multiple groups that each have an aggregate copy of the database.  This is analogous to the replication groups used in HPC-BLAST.  We again partition the database
with:
\begin{verbatim}
$> ./mpiformatdb -N 14 -i nr -o F -p T
\end{verbatim}
\noindent Now, suppose we have requested 6 compute nodes for the parallel job.  In total we will be launching 91 ranks ( \(6 \times (14\,\, \mbox{workers} + 1\,\, \mbox{master} ) + 1\,\, \mbox{super master}\) ).  Now, we create a machine file that describes how the ranks
are distributed to the compute nodes.  On a cluster, this might look like:
\begin{verbatim}
node001:15
node002:15
node003:15
node004:15
node005:15
node006:16
\end{verbatim}
\noindent Note, that the last compute node has one more rank than the other nodes.  This is because mpiBLAST assigns the highest rank number the role of the super master.  Distributing the ranks in this way makes sure that each node has a complete replication group.
Finally, the job can be run with:
\begin{verbatim}
$> micmpiexec -n 91 -machinefile machine.run \
/PATH/TO/WORKDIR/mpiblast --partition-size=15 --use-parallel-write \
--use-virtual-frags -p blastp -d nr -i query.prot -a 1 -o mpiblast.out
\end{verbatim}
\noindent When run with multiple replication groups, mpiBLAST will produce an output file for each group.

\end{document}
